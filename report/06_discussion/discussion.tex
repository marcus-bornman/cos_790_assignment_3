\section{Discussion}
When considering the performance of the two approaches used, one must consider their efficacy in solving the examination timetabling problems from the benchmark set as well as their ability to generalise across different problem instances. In addition, these considerations must be made when comparing the approaches to one another and to other approaches used in the literature.

In comparing the efficacy of the two approaches in solving the given problem instances, there is little evidence to suggest that one approach is more effective. Each approach achieved the better average objective value for 3 of the 6 problem instances while the single-point approach achieved the best overall objective value for 4 of the 6 instances. This indicates relatively similar efficacy in solving the given problems. It is worthwhile to consider, however, that one approach may fair better with larger problem instances but there is insufficient evidence to support this.

When considering the generalisation of the two approaches it would seem that the multi-point search approach does perform slightly better, considering the proportionate standard deviations across problem instances as opposed to the single-point approach. However, it does seem this comes at the cost of performance, with the multi-point search displaying immensely longer runtimes.

Finally, the results of both approaches are not comparable to other approaches in the literature, such as those covered by Pillay \cite{pillay2016review}, due to their inability to find a viable solution. However, an avenue for future research may be to explore an increase in size of the genetic parameters for the multi-point search to discover whether a feasible solution can be found given a larger population or more generations.