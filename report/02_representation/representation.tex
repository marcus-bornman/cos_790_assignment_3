\section{Representation}

\subsection{Nodes}
The possible nodes for each individual, or parse tree, comprise nodes from a function set and terminal set. As per the implementation of the EvoHyp \cite{pillay2017evohyp} toolset, the function set - used for possible internal nodes - contains simple addition, subtraction, multiplication and division arithmetic operators. In the case of the division operator, a value of 1 is returned if the denominator is 0. In addition, arithmetic rules were used so the function set also comprises an \emph{ifelse} operator and
relational operators ($<$, $>$, $<=$, $>=$, $==$ and $!=$).

In terms of the terminal set, there are 3 possible nodes - \emph{A}, \emph{C} and \emph{F} - which represent attributes of a specific city, \emph{n}, in the TSP instance. \emph{A} represents the average distance from \emph{n} to each of its neighbors. \emph{C} represents the distance between \emph{n} and its closest neighbor. \emph{F} represents the distance between \emph{n} and its furthest neighbor. These end nodes, as well as the aforementioned internal nodes, are described in more detail in table \ref{tab:nodes}.

\begin{table}[H]
\resizebox{\textwidth}{!}{\begin{tabular}{|l|l|l|}
\hline
\textbf{Symbol} & \textbf{Arity} & \textbf{Description}                                                                                \\ \hline
+               & 2              & Adds its 2 children and returns the result                                                          \\ \hline
-               & 2              & Subtracts the first child from the second child and returns the result                              \\ \hline
*               & 2              & Multiplies its 2 children and returns the result                                                    \\ \hline
/               & 2              & Divides the 1st child by the 2nd child and returns the result; or, returns 1 if the 2nd child is 0  \\ \hline
\textit{ifelse} & 3              & If the 1st child evaluates to true the 2nd child is returned; otherwise,  the 3rd child is returned \\ \hline
\textless{}     & 2              & True if the 1st child is less than the 2nd child; otherwise, evaluates to false                     \\ \hline
\textgreater{}  & 2              & True if the 1st child is greater than the 2nd child; otherwise, evaluates to false                  \\ \hline
\textless{}=    & 2              & True if the 1st child is less than or equal to the 2nd child; otherwise, evaluates to false         \\ \hline
\textgreater{}= & 2              & True if the 1st child is greater than or equal to the 2nd child; otherwise, evaluates to false      \\ \hline
==              & 2              & True if the 1st child is equal to the 2nd child; otherwise, evaluates to false                      \\ \hline
!=              & 2              & True if the 1st child is not equal to the 2nd child; otherwise, evaluates to false                  \\ \hline
A               & 0              & Returns the average distance from a specific city in the TSP and each of its neighbors              \\ \hline
C               & 0              & Returns the distance from a specific city in the TSP and its closest neighbor                       \\ \hline
F               & 0              & Returns the distance from a specific city in the TSP and its furthest neighbor                      \\ \hline
\end{tabular}}
\caption{Function and Terminal Set}
\label{tab:nodes}
\end{table}

\subsection{Individuals}\label{sec:individuals}
Each chromosome within a population consists of a parse tree for calculating a priority value for each city in the TSP. To clarify, when evaluating the fitness of an individual, the parse tree will be used to calculate a priority value for all cities in the TSP; then, the cities will be visited in order of these priority values and the total distance determined. Given the possible nodes identified in table \ref{tab:nodes}, an example of one such parse tree is depicted in figure \ref{fig:decision_tree}.

% minimum distance between nodes on the same line
\setlength{\GapWidth}{1em}  
% draw with a thick dashed line, very nice looking
\thicklines \drawwith{\dottedline{2}}   
% draw an oval and center it with the rule.  You may want to fool with the
% rule values, though these seem to work quite well for me.  If you make the
% rule smaller than the text height, then the GP nodes may not line up with
% each other horizontally quite right, so watch out.
\newcommand{\gpbox}[1]{\Ovalbox{#1\rule[-.7ex]{0ex}{2.7ex}}}

\begin{figure}[H]
    \centering
    \resizebox{\textwidth}{!}{\begin{bundle}{\gpbox{+}}\chunk{\begin{bundle}{\gpbox{ifelse}}\chunk{\begin{bundle}{\gpbox{$<$}}\chunk{\begin{bundle}{\gpbox{-}}\chunk{\gpbox{A}}\chunk{\gpbox{C}}\end{bundle}}\chunk{\begin{bundle}{\gpbox{-}}\chunk{\gpbox{F}}\chunk{\gpbox{A}}\end{bundle}}\end{bundle}}\chunk{\begin{bundle}{\gpbox{/}}\chunk{\gpbox{A}}\chunk{\gpbox{C}}\end{bundle}}\chunk{\begin{bundle}{\gpbox{/}}\chunk{\gpbox{A}}\chunk{\gpbox{F}}\end{bundle}}\end{bundle}}\chunk{\begin{bundle}{\gpbox{*}}\chunk{\gpbox{A}}\chunk{\begin{bundle}{\gpbox{/}}\chunk{\gpbox{C}}\chunk{\gpbox{F}}\end{bundle}}\end{bundle}}\end{bundle}}
    \caption{Example Individual}
    \label{fig:decision_tree}
\end{figure}

\subsection{Populations}
Of course, a population consists of a collection of the individuals that have been described. The genetic algorithm applied only made use of a single population consisting of 256 individuals. This means that only a single population was evolved and cross-population breeding was not applied.

\section{Initial Population Generation}

As implemented using EvoHyp \cite{pillay2017evohyp}, the initial population was generated using the \emph{grow} method proposed by Koza \cite{koza1992genetic}. At the maximum depth the elements from the terminal set are selected; but, at other levels, elements are selected from both the function and terminal sets, whilst abiding by constraints such that the child of an \emph{ifelse} node must be one of the conditional operator nodes. The maximum tree depth allowed during the initial population generation was a tree depth of 4.